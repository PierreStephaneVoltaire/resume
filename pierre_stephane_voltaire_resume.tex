\documentclass[11pt, letterpaper]{article}

% Packages:
\usepackage[
    ignoreheadfoot, % set margins without considering header and footer
    top=2 cm, % seperation between body and page edge from the top
    bottom=2 cm, % seperation between body and page edge from the bottom
    left=2 cm, % seperation between body and page edge from the left
    right=2 cm, % seperation between body and page edge from the right
    footskip=1.0 cm, % seperation between body and footer
% showframe % for debugging
]{geometry} % for adjusting page geometry
\usepackage{titlesec} % for customizing section titles
\usepackage{tabularx} % for making tables with fixed width columns
\usepackage{array} % tabularx requires this
\usepackage[dvipsnames]{xcolor} % for coloring text
\definecolor{primaryColor}{RGB}{0, 0, 0} % define primary color
\usepackage{enumitem} % for customizing lists
\usepackage{fontawesome5} % for using icons
\usepackage{amsmath} % for math
\usepackage[
    pdftitle={Pierre Voltaire's CV},
    pdfauthor={Pierre Voltaire},
    pdfcreator={LaTeX with RenderCV},
    colorlinks=true,
    urlcolor=primaryColor
]{hyperref} % for links, metadata and bookmarks
\usepackage[pscoord]{eso-pic} % for floating text on the page
\usepackage{calc} % for calculating lengths
\usepackage{bookmark} % for bookmarks
\usepackage{lastpage} % for getting the total number of pages
\usepackage{changepage} % for one column entries (adjustwidth environment)
\usepackage{paracol} % for two and three column entries
\usepackage{ifthen} % for conditional statements
\usepackage{needspace} % for avoiding page brake right after the section title
\usepackage{iftex} % check if engine is pdflatex, xetex or luatex

% Ensure that generate pdf is machine readable/ATS parsable:
\ifPDFTeX
\input{glyphtounicode}
\pdfgentounicode=1
\usepackage[T1]{fontenc}
\usepackage[utf8]{inputenc}
\usepackage{lmodern}
\fi

\usepackage{charter}

% Some settings:
\raggedright
\AtBeginEnvironment{adjustwidth}{\partopsep0pt} % remove space before adjustwidth environment
\pagestyle{empty} % no header or footer
\setcounter{secnumdepth}{0} % no section numbering
\setlength{\parindent}{0pt} % no indentation
\setlength{\topskip}{0pt} % no top skip
\setlength{\columnsep}{0.15cm} % set column seperation
\pagenumbering{gobble} % no page numbering

\titleformat{\section}{\needspace{4\baselineskip}\bfseries\large}{}{0pt}{}[\vspace{1pt}\titlerule]

\titlespacing{\section}{
% left space:
    -1pt
}{
% top space:
    0.3 cm
}{
% bottom space:
    0.2 cm
} % section title spacing

\renewcommand\labelitemi{$\vcenter{\hbox{\small$\bullet$}}$} % custom bullet points
\newenvironment{highlights}{
\begin{itemize}[
topsep=0.10 cm,
parsep=0.10 cm,
partopsep=0pt,
itemsep=0pt,
leftmargin=0 cm + 10pt
]
}{
\end{itemize}
} % new environment for highlights


\newenvironment{highlightsforbulletentries}{
\begin{itemize}[
topsep=0.10 cm,
parsep=0.10 cm,
partopsep=0pt,
itemsep=0pt,
leftmargin=10pt
]
}{
\end{itemize}
} % new environment for highlights for bullet entries

\newenvironment{onecolentry}{
\begin{adjustwidth}{
0 cm + 0.00001 cm
}{
0 cm + 0.00001 cm
}
}{
\end{adjustwidth}
} % new environment for one column entries

\newenvironment{twocolentry}[2][]{
\onecolentry
\def\secondColumn{#2}
\setcolumnwidth{\fill, 4.5 cm}
\begin{paracol}{2}
}{
\switchcolumn \raggedleft \secondColumn
\end{paracol}
\endonecolentry
} % new environment for two column entries

\newenvironment{threecolentry}[3][]{
\onecolentry
\def\thirdColumn{#3}
\setcolumnwidth{, \fill, 4.5 cm}
\begin{paracol}{3}
{\raggedright #2} \switchcolumn
}{
\switchcolumn \raggedleft \thirdColumn
\end{paracol}
\endonecolentry
} % new environment for three column entries

\newenvironment{header}{
\setlength{\topsep}{0pt}\par\kern\topsep\centering\linespread{1.5}
}{
\par\kern\topsep
} % new environment for the header

\newcommand{\placelastupdatedtext}{% \placetextbox{<horizontal pos>}{<vertical pos>}{<stuff>}
\AddToShipoutPictureFG*{% Add <stuff> to current page foreground
\put(
\LenToUnit{\paperwidth-2 cm-0 cm+0.05cm},
\LenToUnit{\paperheight-1.0 cm}
){\vtop{{\null}\makebox[0pt][c]{
\small\color{gray}\textit{Last updated in September 2024}\hspace{\widthof{Last updated in September 2024}}
}}}%
}%
}%

% save the original href command in a new command:
\let\hrefWithoutArrow\href

% new command for external links:


\begin{document}
\newcommand{\AND}{\unskip
\cleaders\copy\ANDbox\hskip\wd\ANDbox
\ignorespaces
}
\newsavebox\ANDbox
\sbox\ANDbox{$|$}

\begin{header}
\fontsize{15 pt}{15 pt}\selectfont Pierre-Stéphane Voltaire

\vspace{2 pt}

\normalsize
\kern 5.0 pt%
\mbox{\hrefWithoutArrow{mailto:psvoltaire96@gmail.com}{psvoltaire96@gmail.com}}%
\kern 5.0 pt%
\AND%
\kern 5.0 pt%
\mbox{\hrefWithoutArrow{tel:+1-905-616-0236}{+1(905)-616-0236}}%
\kern 5.0 pt%
\AND%
\kern 5.0 pt%
\mbox{\hrefWithoutArrow{https://psvoltaire.ca/}{psvoltaire.ca}}%
\kern 5.0 pt%
\AND%
\kern 5.0 pt%
\mbox{\hrefWithoutArrow{https://www.linkedin.com/in/pierre-stephane-voltaire/}{linkedin.com/in/pierre-stephane-voltaire}}%
\kern 5.0 pt%
\AND%
\kern 5.0 pt%
\mbox{\hrefWithoutArrow{https://github.com/PierreStephaneVoltaire}{github.com/PierreStephaneVoltaire}}%
\end{header}

\vspace{2 pt - 0.3 cm}






\section{Summary}

\begin{onecolentry}
\begin{highlightsforbulletentries}

Senior DevOps Engineer with 7 years of experience architecting cloud-native infrastructure, implementing\textbf{ Infrastructure as Code}, and building scalable \textbf{CI/CD} pipelines. Improved system uptime by automating manual deployments and migrating workloads to high-availability architectures. Collaborate with development teams to identify and resolve application bottlenecks, enabling cloud-friendly refactoring initiatives. Strong coding background with proficiency in \textbf{Python}, \textbf{Go}, \textbf{JavaScript} and \textbf{Java} combined with extensive experience with Infrastructure as Code using \textbf{Terraform}, \textbf{AWS CDK} and \textbf{CloudFormation}. Hands-on experience across \textbf{AWS} infrastructure including serverless technologies (\textbf{Lambda, Step Functions}), traditional compute (\textbf{EC2}), and container orchestration (\textbf{ECS}, \textbf{Kubernetes}), with ongoing \textbf{Azure} experience.







\end{highlightsforbulletentries}
\end{onecolentry}





\section{Skills}

\begin{onecolentry}
\textbf{Programming \& Scripting:} Java,
TypeScript, JavaScript, Golang, Python, Bash, Powershell, SQL
\end{onecolentry}

\vspace{0.05 cm}
\begin{onecolentry}
\textbf{Development library \& Frameworks:} Spring Boot, Flask, Gin, Vue.js, Angular, ReactJS
\end{onecolentry}

\vspace{0.05 cm}
\begin{onecolentry}
\textbf{CI\//CD:} CodeDeploy, CodePipeline, CodeBuild, CircleCI, GitHub Actions, Jenkins, GitLab CI, ArgoCD, Hashicorp Vault \end{onecolentry}

\vspace{0.05 cm}
\begin{onecolentry}
\textbf{Infrastructure as code:} Terraform, Terragrunt, AWS CDK, Cloudformation \end{onecolentry}

\vspace{0.05 cm}
\begin{onecolentry}
\textbf{Containerization \& Orchestration:} Docker, Podman, Docker Compose, Helm,
ECS, Fargate, Kubernetes

\end{onecolentry}
\vspace{0.05 cm}

\begin{onecolentry}
\textbf{Security \& Compliance:} Wiz, Snyk, SonarQube


\end{onecolentry}

\vspace{0.05 cm}
\begin{onecolentry}
\textbf{Monitoring \& Observability:} Datadog APM, CloudWatch


\end{onecolentry}

\vspace{0.05 cm}
\begin{onecolentry}
\textbf{Identity Provider:} AWS Cognito, Microsoft Entra, Keycloak


\end{onecolentry}

\vspace{0.05 cm}

\begin{onecolentry}
\textbf{Source control management:} GIT, TFS, SVN


\end{onecolentry}
\vspace{0.05 cm}

\begin{onecolentry}
\textbf{Documentation \& Project Management:} Jira, Confluence, GitHub/GitLab Wiki, Markdown, Draw.io, Scrum, Kanban


\end{onecolentry}







\section{Experience}

\begin{twocolentry}{
Mar 2021 – Present
}
\textbf{Senior Platform Engineer, Mitsubishi UFJ Financial Group Investor Service}  -- Toronto, ON\end{twocolentry}

\vspace{0.10 cm}
\begin{onecolentry}
\begin{highlights}
\item Developed a FastAPI (Python) GitLab backup solution that protects $\sim$ 1000 repositories against extended outages and accidental deletions, with deployment handled through ArgoCD and Kustomize for flexible environment configuration.
\item Automated EKS upgrades for 20+ clusters using GitLab CI and Terraform, eliminating manual work and reducing \$6K/month in extended support costs.
\item Reduced production deployment time by 94\% by building CI/CD pipelines using GitLab CI, CodeDeploy, and custom Python, Bash, and Powershell scripts.
\item Developed secure and efficient processes and tools for managing secrets and injecting application configurations using Jenkins and Lambda functions in Python and Node.js.
\item Implemented security measures during the migration of Java and .NET web applications to AWS, aligning with SOC 2 compliance by enhancing network security with HTTPS and ACM certificates, encrypting EFS storage with AWS KMS, and utilizing IAM for access control enforcement.
\item Provisioned new cloud environments for greenfield applications using a variety of IaC tools, including CloudFormation, Terraform, and AWS CDK.
\item  Migrated repositories from Visual Studio Team Foundation Server (TFS) and TortoiseSVN to GitLab, ensuring a smooth transition and effective version control management.
\item  Managed artifacts across multiple platforms, including Sonatype Nexus, GitLab, and Amazon ECR, overseeing Java, .NET, and Python artifacts as well as Docker images.
\item Enhanced monitoring capabilities by integrating Datadog APM across cloud-native applications, improving visibility into performance issues, and accelerating troubleshooting.
% \item Migrated 6 legacy Maven and Dotnet applications to AWS within 2 years, provisioning cloud native resources, optimizing system performance and automating build, test, and deployment processes
\end{highlights}
\end{onecolentry}



\vspace{0.2 cm}

\begin{twocolentry}{
Aug 2020 – Mar 2021
}
\textbf{FullStack Developer, Airborne App}  -- Oakville, ON\end{twocolentry}

\vspace{0.10 cm}
\begin{onecolentry}
\begin{highlights}
\item Developed features on existing Golang services that interact with multiple AWS services
% \item Integrated user-facing components on frontend apps with React, TypeScript, and SASS within 5 days, improving overall user experience and app performance
\item Developed CI pipelines leveraging GitHub Actions and Terraform to automatically lint, execute unit and integration tests for incoming pull requests, and reduce the introduction of bugs in production
\item Developed an ephemeral testing environment pipeline using GitHub actions, terraform and S3, accelerating frontend validation processes and streamlining collaboration between developers and product owners.
\end{highlights}
\end{onecolentry}



\vspace{0.2 cm}

\begin{twocolentry}{
May 2019 – Apr 2020
}
\textbf{FullStack Developer, EllisDon}  -- Mississauga, ON\end{twocolentry}

\vspace{0.10 cm}
\begin{onecolentry}
\begin{highlights}
\item \textbf{Developed and maintained full-stack applications} using ReactJS, Spring Boot (Java), Express (Node.js), NestJS (TypeScript), and MongoDB, ensuring seamless feature integration and enhanced code maintainability
\item \textbf{Implemented automated CI pipeline}s for unit testing and code quality enforcement using linting tools, improving project maintainability and reducing technical debt
% \item Implemented tests using Junit and Jest to boost code coverage and enhanced code quality in frontend and backend applications.
\end{highlights}
%      \vspace{0.2 cm}

% \begin{twocolentry}{
%     Sep 2018 - Mar 2019
% }
%     \textbf{Mobile Web Developer, Sheridan College Research}  -- Oakville, ON\end{twocolentry}

% \vspace{0.10 cm}
% \begin{onecolentry}
%     \begin{highlights}
%         \item Built a data aggregation system using Vue.js, Python, and Flask to streamline the analysis of data from third-party APIs and website content.
%         \item Provisioned and managed databases and virtual machines in DigitalOcean to host applications, significantly reducing the feedback loop between developers and stakeholders.

%   \end{highlights}
\end{onecolentry}


\section{Education}




\begin{twocolentry}{
Sept 2016 – Apr 2020

}
\textbf{Sheridan College}, Bachelor of Computer Science – Mobile Computing\end{twocolentry}

\vspace{0.10 cm}

\begin{twocolentry}{
Sept 2014 – Aug 2016
}
\textbf{Sheridan College}, Computer Systems Technician – Software Engineering\end{twocolentry}

\vspace{0.10 cm}



\end{document}